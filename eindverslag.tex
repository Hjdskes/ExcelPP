\documentclass[a4paper,11pt]{article}
\usepackage[T1]{fontenc}
\usepackage[utf8]{inputenc}
\usepackage{lmodern}
\usepackage{amssymb}
\usepackage{color}
\newcommand{\tab}{\hspace*{2em}}

\title{Excel++}
\author{Bernd Kreynen 4331842\\
		Dave Grund 4291999\\
		Gerlof Fokkema 4257286\\
		Jente Hidskes 4335732\\
		Joris Lamers 4233042\\
		Skip Lentz <studentnummer>\\
	   }

\begin{document}

\begin{titlepage}
\maketitle
\thispagestyle{empty} %geen page numbering op opening pagina
\end{titlepage}

\newpage\section{Algemeen: hoe het project is verlopen}\mbox{} \\
Over het algemeen is het project goed verlopen. Hier en daar zaten we
een aantal keren vast, moest er even goed worden nagedacht over de juiste
aanpak of ging er iets mis met GitHub, maar niks dat niet overkomen kon
worden. Iedereen was tevreden met zijn taak, overleg plegen en planningen
maken ging altijd goed en ook de communicatie met de student assistent
was leuk. Wij zijn dan ook erg tevreden met het eindresultaat en de groep
waarin wij zaten.\\

Van begin af aan was er in onze groep een duidelijke taakverdeling: de
groep van zes is opgesplitst in tweetallen met ieder hun eigen taak. Zo
werkte één tweetal aan de XML-reader en -writer en de formules, werkte
een ander tweetal aan de parser voor de formules en de implementatie van
Cellen en werkte het laatste tweetal aan de GUI. Doordat er zo’n duidelijke
taakverdeling was, werd het hele project overzichtelijk en was het makkelijk
om overleg te voeren. Er werd eerst overleg gepleegd binnen het tweetal en
daarna over de conclusies van de andere tweetallen. Zo zijn eindeloze dis-
cussies en het tegelijkertijd implementeren van dezelfde feature voorkomen.\\

Helemaal in het begin, bij de eerste meeting, is er overleg gepleegd
over wat voor features Excel++ moest hebben. Daarna is de taakverdel-
ing gemaakt, onder andere gebaseerd op ervaring die sommige teamleden al
hadden: Gerlof heeft al \textit{"real world"} ervaring met programmeren dus zou
hij de formule parser en de implementatie op zich nemen, omdat dit een
kritiek onderdeel van elk Spreadsheet programma is. Jente had al ervaring
met GUI’s maken (GTK+ toolkit op Linux), dus nam hij dit onderdeel voor
zijn rekening. Tot slot had Joris al ervaring met XML bestanden, omdat hij
webdeveloper is naast student.\\

De gemaakte planningen waren eenvouding: \textit{"het grondwerk voor fea-
ture X is nu gelegd, dus volgende week is dat op zijn minst basaal geïmple-
menteerd"}. Zo zijn we iedere week bij elkaar gekomen om te kijken waar
we nu waren, wat er nog moest gebeuren en welke features nu aan de beurt
waren. Logischerwijs moest er met sommige features gewacht worden totdat
anderen waren gedaan, maar dit was geen probleem. De hele groep hield
zich aan de planning en er is nooit een periode geweest dat het ene tweetal
moest wachten op het andere. Dit had tot gevolg dat iedereen op zijn eigen
tempo door kon werken en dat steeds elke planning zoals gepland doorlopen
werd.\\

Overleg plegen gebeurde voornamelijk via Facebook; hier hebben bij een
speciale, private pagina op aangemaakt voor onze groep. Meer richting het
einde werd er ook gebruikt gemaakt van de Issues op GitHub, maar deze
diende voornamelijk als bugtracker en niet voor het overleggen over bijvoor-
beeld welke toolkit er gekozen moest worden. Er is voor Facebook gekozen
omdat iedereen hier toch wel dagelijks even naar kijkt en dit bij bijvoorbeeld
email niet het geval is. Als het GUI-team een nieuwe methode nodig had
van de klasse Cel of een wijziging in het TableModel, is het erg fijn als dit
nog diezelfde middag gedaan kan worden. Via Facebook was dit mogelijk.
Facebook bleek ook handig voor het terugkijken op eerder gemaakte beslissin-
gen: alles is handig op chronologische volgorde terug te vinden, met de re-
acties van groepsleden erbij. Zoals gezegd is er ook gebruik gemakt van de
Issues pagina van GitHub als bugtracker. Deze pagina is erg handig voor
het centraal bijhouden van bugs of nog missende features. Ook hier is wel
enig overleg gepleegd, wat wel resulteerde tot twee plaatsen van overleg en
dus soms wel enig zoekwerk naar de juiste informatie. Een leuke bijkomend-
heid van de Issues is de drang die je hebt om het aantal geopende Issues zo
laag mogelijk te houden: zo word je nog gemotiveerder om bugs te squashen!\\

Het gebruik van Git was niet alleen handig vanwege GitHub’s Issues
pagina. Er is in onze groep veelvuldig gebruik gemaakt van branches, van
Git’s rebase mogelijkheid en om simpelweg een commit te kunnen reverten.
Tot twee keer toe was er een probleem met de master repository, maar mede
dankzij de lokale kopieën zijn we altijd in staat geweest om dit op te lossen.
De mogelijkheid om \textit{diff’s} van elke commit na te kijken is ook erg handig
om te zien wat er precies is veranderd. Dit biedt de mogelijkheid om te zien
of iets niet beter op een andere manier gedaan had kunnen worden. Er kan
dus geconcludeerd worden dat het gebruik van versiebeheer ons zeker heeft
geholpen!\\

\newpage\section{Ontwerpproces}
\end{document}
